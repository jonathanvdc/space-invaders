\documentclass[10pt,a4paper]{article}
\usepackage[utf8]{inputenc}
\usepackage{amsmath}
\usepackage{amsfonts}
\usepackage{amssymb}
\usepackage{graphicx}
\usepackage[dutch,english]{babel}
\usepackage[obeyspaces]{url}
\usepackage[multiple]{footmisc}

\author{Jonathan Van der Cruysse}
\title{Rapport: Project Gevorderd Programmeren}

\begin{document}
	
\newcommand{\classname}[1]
{
	\selectlanguage{english}
	\textsf{#1}
	\selectlanguage{dutch}
}
	
\maketitle

\section{Algemeen}
Tijdens het maken van dit project 
-- \emph{space invaders} -- had ik de volgende
doelstellingen voor ogen: \emph{hackability}, 
Newtoniaanse fysica, en natuurlijk ook de \emph{space setting}
van het originele spel. 

Met het begrip \emph{hackable} bedoel ik dat elke gebruiker arbitrair
complexe sc\`enes kan opbouwen -- en die daarna spelen --, zonder een 
letter aan de programmacode te veranderen. Dit beperkt zich niet enkel
tot het cosmetische of een initi\"ele opstelling van invaders: het is 
mogelijk een volledige \emph{timeline} op te stellen, die precies
bepaalt wat er gebeurt, in chronologische volgorde. 
Zelfs gebeurtenissen zoals een spel `winnen' of `verliezen' worden 
vertolkt door zo'n \emph{timeline}. In die zin is
het programma zelf eerder gelijkend op een vorm van \emph{engine}, die 
scenario's tot uitvoering brengt.

Voor bewegingen beroept het spel zich op Newton's eerste wet: 
\emph{entities} die eenmaal in beweging gebracht zijn, zullen niet
uit zichzelf tot stilstand komen. Alle \emph{entities} zijn hier gevoelig
aan. Zo kunnen ook projectielen afgebogen worden door een object dat 
zwaartekracht (of anti-zwaartekracht) uitstraalt.

De setting van het spel in de ruimte hangt grotendeels samen met het 
voorgenoemde principe. Door het ontbreken van fenomenen zoals wrijving
in de ruimte, gaan deze twee dan ook goed samen.


\section{Programma-structuur}
\subsection{Algemene onderverdeling}
De \emph{source files} van het programma zijn onderverdeeld in volgende 
categorie\"en:
\begin{itemize}
\item \textbf{Model:} Dit onderdeel van \emph{space invaders} bevat de staat van het 
spel zelf. Ik heb opzettelijk de verantwoordelijkheden van het \emph{model} minimaal 
gehouden: \emph{entities} beheren hun eigen positie en levensduur, maar interageren 
niet direct met andere \emph{entities}. 
(directory: \path{src/model/})

\item \textbf{View:} Representeert de visuele staat van het spel. \emph{Renderables} 
staan gewoonlijk los van het \emph{model}: ze vormen eerder een
`gereedschapskist' van herbruikbare, \emph{single-purpose} klassen die samengeplaatst 
kunnen worden in een boomstructuur om complexere figuren te vormen. \classname{PathOffsetRenderable} en \classname{TransformedRenderable} worden gebruikt om 
de positie en ori\"entatie van \emph{entities} te synchroniseren met hun 
overeenkomstige \emph{renderables}. 

Dit is niet gebaseerd op een `ingebakken' 
ondersteuning van \emph{model} elementen, maar wel op een arbitraire 
functie, die opgeroepen wordt om de positie en transformatie van de \emph{renderable 
tree} te bepalen. De \emph{view} is op die manier niet afhankelijk van de overige
onderdelen van \emph{space invaders}, wat het geschikt maakt voor gebruik in 
een ander spel. 
(directory: \path{src/view/})

\item \textbf{Controller:} \emph{Controllers} zijn enerzijds verantwoordelijk voor 
de interacties tussen verschillende \emph{entities}, en anderzijds voor het 
`dirigeren' van individuele \emph{entities}. Zo wordt zwaartekracht geregeld door
een \classname{GravityController}, terwijl het schip van een \emph{invader} geleid wordt 
door een \classname{PathController}. 
(directory: \path{src/controller/})

\item \textbf{Timeline:} Naast de klassieke \emph{model-view-controller} 
combinatie steunt dit spel ook op de vierde pilaar van de \emph{timeline}, die 
bestaat uit \emph{events} die tijdens het spel (chronologisch) aan bod komen, en
volledige controle over de sc\`ene hebben. Het toevoegen van \emph{invaders} of 
obstakels, en daarnaast ook het tonen van UI-elementen zoals tekst, zijn
typische use-cases voor \emph{events}.
(directory: \path{src/timeline/})

\item \textbf{Parser:} Deze component wordt gebruikt om het spel te 
initializeren met data uit een XML \emph{scene description} bestand.

Indien dit syntactische of semantische fouten bevat, vangt
de parser ze op door middel van een \emph{exception}, die door het
\emph{entry point} van het programma omgezet wordt in console-uitvoer
voor de gebruiker.
(directory: \path{src/parser/})

\item \textbf{Top-level helpers:} Helper klassen die nuttig zijn voor de 
voorgenoemde onderdelen, zoals \classname{Event}, \classname{RandomGenerator} en \classname{Stopwatch}. 
Anderzijds zitten hier ook klassen tussen die de voorgenoemde componenten
gebruiken, en samen plaatsen, zoals \classname{Scene} - een type object
dat alle data gerelateerd aan een `spelbord' bijhoudt.
(directory: \path{src/})

\end{itemize}

\subsection{Design patterns}

De implementatie van het spel bedient zich van deze \emph{design patterns}, zoals vereist in de specificatie:

\begin{itemize}
	
\item \textbf{Model-View-Controller:} Zoals reeds vermeld in de 
projectstructuur,
bedient deze \emph{space invaders} zich van het \emph{model-view-controller}
patroon. Daarnaast is ook nog de \emph{timeline} toegevoegd, die het op een
nog macroscopischer niveau werkt.

\item \textbf{Singleton:} Overeenkomstig met de specificatie, zijn 
de \classname{RandomGenerator} en \classname{Stopwatch} singletons: er bestaat maar \'e\'en instantie van deze klassen.

\item \textbf{Observer:} Dit \emph{design pattern} wordt ge\"implementeerd
door het \classname{Event} \emph{class template}: dit stelt een 
ge\"idealiseerd \emph{observable} object voor, met als 
observers -- of \emph{event handlers} -- 
\classname{std::function} instanties. 
Zo'n \classname{Event} kan dan als field in een klasse opgenomen worden. 

\classname{Scene} maakt hier dankbaar gebruik van: de voorgenoemde 
klasse bevat zowel een \classname{Game} en \classname{GameRenderer}. 
De laatste bevat alle \emph{renderables}, de eerste alle \emph{entities}.
Wanneer een \emph{entity} verwijderd wordt, vangt de \classname{Scene}
een bericht op uit het \classname{Game} object, en wordt de overeenkomstige
\emph{renderable} meteen ook verwijderd. Voor controllers is dit niet nodig:
zij bepalen immers zelf wanneer ze verwijderd dienen te worden.


\end{itemize}

\subsection{Flexibel ontwerp}
Het toewijzen van gedrag aan \emph{controllers} in plaats van 
\emph{entities} heeft als gevolg dat de klassen-hi\"erarchie van 
die laatsten
\footnote{Inheritance-diagram van \classname{Entity}: \url{Entity_inheritance.png} (in bijlage)}\footnote{Inheritance-diagram van \classname{IController}: \url{Controller_inheritance.png} (in bijlage)}
 eerder beperkt blijft. Zowel \emph{player} als \emph{invader}
schepen worden door een \classname{ShipEntity} voorgesteld. 
Het verschil tussen de twee zit hem immers in de \emph{controllers} die aan 
de schepen toegewezen worden. Dit vindt ook zijn weerklank in 
\emph{scene descriptions}: het is mogelijk om bepaalde \emph{controllers}, 
zoals een \classname{GravityController} aan eenderwelke entity toe te 
voegen. Typische toepassingen hiervan zijn een 
\emph{`gravity well'}, `magnetische kogels' en een `schild' dat 
\emph{entities} afstoot.

\subsection{Libraries}
Dit project steunt op de \texttt{SFML} \emph{library} -- en gaat ervan uit
dat deze als \emph{package} aanwezig is --, en includeert 
\texttt{tinyxml2}. Dat laatste wordt door \url{CMakeLists.txt} als 
gecompileerd, en gelinkt met de \emph{executable}.

\section{Uitbreidingen}

Deze implementatie omvat ook enkele uitbreidingen, bovenop het basispakket:

\begin{itemize}

\item \textbf{Audio:} Zowel geluidseffecten als achtergrondmuziek kunnen 
in de vorm van \emph{timeline events} afgespeeld en gestopt worden.

\item \textbf{Verschillende soorten invaders:} Het XML 
\emph{scene description} formaat van deze \emph{space invaders} laat
de gebruiker toe zowel de fysieke eigenschappen van de 
\emph{entity} van \emph{invaders} -- of het schip van de speler --, 
als de \emph{renderable} die daarmee geassocieerd is
aan te passen.

\item \textbf{Geanimeerde sprites:} naast statische sprites, is het
ook mogelijk hun geanimeerde soortgenoten als \emph{renderable} te 
gebruiken.

\item \textbf{Timeline:} Meer dan een simpele uitbreiding -- 
maar toch het vermelden waard: de hierboven
reeds toegelichte \emph{timeline} is een van de pijlers waarop het programma
steunt.

\item \textbf{Verschillende timelines:} Een uitbreiding op een 
andere uitbreiding: naast de \emph{main timeline} van een 
\emph{scene description}, heeft elke \emph{entity} zijn eigen 
\emph{timeline}. Deze begint met het cre\"eren van een \emph{entity}, 
en eindigt wanneer die \emph{entity} vernietigd wordt. Het is mogelijk
en veelvoorkomend om daaraan extra \emph{timeline events} toe te voegen.
Dit maakt het praktisch om bijvoorbeeld een geluidseffect af te spelen
wanneer een \emph{invader} wordt vernietigd. Eventueel kan ook een 
\emph{controller} een event starten, bijvoorbeeld als een \emph{entity}
de sc\`ene verlaat, of een bepaalde sectie van de sc\`ene binnengaat.

\item \textbf{Extra timeline events:}
\begin{itemize}
	
\item \textbf{Condities en flags:} Bepaalde \emph{timeline events} kunnen
conditioneel gemaakt worden: \'e\'en van twee \emph{timelines} worden 
uitgevoerd op basis van een \emph{boolean flag}. 
Mogelijk interessant is dat deze conditie op elke frame 
opnieuw ge\"evalueerd wordt. Deze aanpak blijkt in praktijk 
verassend goed te werken: een `typische' \emph{timeline}, waarin
invaders gespawned worden, wordt uitgevoerd tot het schip van de speler
vernietigd wordt. Bij het \emph{destroyed event} van die laatste 
wordt een \emph{flag} op ``false'' gezet. Vervolgens `bedenkt' de conditie 
zich bij het opstellen van
de volgende \emph{frame}, wordt de huidige 
\emph{timeline} be\"eindigd, en wordt een andere \emph{timeline} opgestart,
die door middel van wat tekst uitlegt dat de speler verloren heeft.

\item \textbf{Loops:} Gelijkaardig aan het voorgaande, bestaan er ook
\emph{loop timeline events}, die hun \emph{child event} een eindig of
oneindig aantal keer herhalen.

\item \textbf{Deadlines:} \emph{timelines} kunnen ook stopgezet worden na
een bepaalde tijdsduur. Dit houdt opzettelijk geen rekening met het
al dan niet klaar-zijn van de stopgezette \emph{timeline}.

\end{itemize}

\end{itemize}

\section{Praktisch}

Het \url{run.sh} \emph{shell script} compileert het 
\texttt{space-invaders} programma, en
voert dit meteen uit op het argument dat meegegeven wordt aan dit
script.


\end{document}